\subsection*{1A~演習問題第21問}
自然数$n$に関する三つの条件$p,\:q,\:r$を次のように定める。
\begin{align*}
    p&~:~nは4の倍数である\\
    q&~:~nは6の倍数である\\
    r&~:~nは24の倍数である
\end{align*}
条件$p,\:q,\:r$の否定をそれぞれ$\bar{p},\:\bar{q},\:\bar{r}$で表す。\\
条件$p$を満たす自然数全体の集合を$P$,\:条件$q$を満たす自然数全体の集合を$Q$,\:条件$r$を満たす自然数全体の集合を$R$とする。\\
自然数全体の集合を全体集合とし,\:集合$P,\:Q,\:R$の補集合をそれぞれ$\bar{P},\:\bar{Q},\:\bar{R}$で表す。このとき,\:次の問いに答えよ。
\begin{enumerate}[left=0pt,label=(\arabic*)]
    \item 
        $P\cap Q$に属する最小の自然数$a$を求めよ。
    \item 
        $a~\square~  R$である。$\square$に当てはまる記号を答えよ。
\end{enumerate}
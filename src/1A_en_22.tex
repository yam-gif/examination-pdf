\subsection*{1A~演習問題第22問}
全体集合を$1$桁の自然数全体とするとき,\:$2$つの集合$A,\:B$を次のように定める。
\begin{align*}
    A=\{x \mid xは1桁の素数 \},\:B=\{x \mid xは1桁の3の倍数\}
\end{align*}
このとき,\:次の問いに答えよ。ただし,\:集合$X$に対して集合$\bar{x}$は集合$X$の補集合を表す。
\begin{enumerate}[left=0pt,label=(\arabic*)]
    \item 
        $A,\:B$を要素を書き並べる形で表せ。
    \item 
        $A\cap B,\:A\cup B,\:\bar{A},\:\bar{B},\:\bar{A}\cap B,\:A\cup\bar{B}$を要素を書き並べる形で表せ。
\end{enumerate}